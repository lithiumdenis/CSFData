\documentclass[10pt]{article} % Документ принадлежит классу article, а также будет печататься в 12 пунктов.
\usepackage{array} % Для титульника
\usepackage{ucs}
\usepackage[utf8x]{inputenc} % Включаем поддержку UTF8
\usepackage[russian]{babel}  % Включаем пакет для поддержки русского языка
\usepackage[left=2cm,right=2cm, top=2cm,bottom=2cm,bindingoffset=0cm]{geometry} % Отступы по краям страницы
\usepackage{amssymb,amsfonts,amsmath,mathtext,cite,enumerate,float} % Математические штуки
\usepackage{cmap} % чтобы работал поиск по PDF
\usepackage{graphicx} % для вставки картинок
\usepackage{epstopdf} 	

\usepackage{wrapfig} % Обтектание картинок текстом
\usepackage{caption}
\usepackage{subcaption}
 
%  для гиперссылок
\usepackage{xcolor}
\usepackage{hyperref}
\definecolor{linkcolor}{HTML}{191970} % цвет ссылок
\definecolor{urlcolor}{HTML}{191970} % цвет гиперссылок
\hypersetup{pdfstartview=FitH,  linkcolor=linkcolor,urlcolor=urlcolor, colorlinks=true}

\usepackage{pscyr} % Нормальные шрифты
\usepackage{setspace} % Для отступов между строк
%Это для формирования листингов
%%%%%%%%%%%%%%%%%%%%%%%%Листинги на MATLAB
\usepackage{listings}
\usepackage{color} %red, green, blue, yellow, cyan, magenta, black, white
\definecolor{mygreen}{RGB}{28,172,0} % color values Red, Green, Blue
\definecolor{mylilas}{RGB}{170,55,241}

\lstset{language=Matlab,
	breaklines=true,%
	morekeywords={matlab2tikz},
	keywordstyle=\color{blue},%
	morekeywords=[2]{1}, keywordstyle=[2]{\color{black}},
	identifierstyle=\color{black},%
	stringstyle=\color{mylilas},
	commentstyle=\color{mygreen},%
	showstringspaces=false,%without this there will be a symbol in the places where there is a space
	numbers=left,%
	numberstyle={\tiny \color{black}},% size of the numbers
	numbersep=9pt, % this defines how far the numbers are from the text
	emph=[1]{for,end,break},emphstyle=[1]\color{blue}, %some words to emphasise
	inputencoding=cp1251
	%emph=[2]{word1,word2}, emphstyle=[2]{style},    
}

%%%%%%%%%%%%%%%%%%%%%%%%%%%%%%%%%


\begin{document} % Начало документа

\title{{\large Распознавание образов. Лабораторная работа №5.} \\
	\textbf{\textquotedblleft Распознавание образов в условиях параметрической неопределенности и обучения с учителем\textquotedblright}.\\
	{\large Вариант 9}
	}
\date{}
\author{\textit{Выполнил}: студент 4 курса, группы 6.1 \\
	Суходолов Денис}
        
\maketitle

\begin{spacing}{1} % Для отступов между строк
\section*{Цель работы}
Синтезировать алгоритмы распознавания  образов, описываемых
гауссовскими распределениями с неизвестными параметрами. Оценить
неизвестные параметры распределений на основе объектов обучающих
выборок методом максимального правдоподобия и максимума апостериорной
вероятности. Исследовать синтезированные алгоритмы распознавания с точки
зрения ожидаемых потерь и ошибок в зависимости от объема обучающей
выборки.
\section*{Исходные данные}
\textbf{Значения параметров:}
$$ m_1\;=\;[0\;-1\;\;\;\;2],\;m_2\;=\;[2\;\;2\;\;1],\;C\;=\;[4\;-1\;-1;\;-1\;\;\;4\;-1;-1\;-1\;\;\;\;4].$$
\textbf{Априорные вероятности гипотез:}
~\\
\begin{tabular}{| l | c | c |}
	\hline			
	Случай & $p(w_1)$ & $p(w_2)$ \\
	\hline	
	$p(w_1) > p(w_2)$ & 0.8 & 0.2 \\
	\hline	
	$p(w_1) = p(w_2)$ & 0.5 & 0.5 \\
	\hline  
	$p(w_1) < p(w_2)$ & 0.2 & 0.8 \\
	\hline  
\end{tabular}
~\\
\section*{Код для оценок неизвестных параметров (МП и МАВ)}
\lstinputlisting[basicstyle=\ttfamily\footnotesize ]{param.m} 
\section*{Код расчёта разделяющих функций и вероятностей ошибок распознавания}
\lstinputlisting[basicstyle=\ttfamily\footnotesize ]{param2.m} 
\section*{Графики  значений  элементов  теоретической  и экспериментальной матриц вероятностей ошибок}
\begin{figure}[h]
	\centering
	\begin{subfigure}{.5\textwidth}
		\centering
		\includegraphics[width=1.0\linewidth]{1.eps}
		\caption{$p(w_1) > p(w_2)$}
		%\label{fig:sub1}
	\end{subfigure}%
	\begin{subfigure}{.5\textwidth}
		\centering
		\includegraphics[width=1.0\linewidth]{2.eps}
		\caption{$p(w_1) = p(w_2)$}
		%\label{fig:sub2}
	\end{subfigure}
	\begin{subfigure}{.5\textwidth}
		\centering
		\includegraphics[width=1.0\linewidth]{3.eps}
		\caption{$p(w_1) < p(w_2)$}
		%\label{fig:sub2}
	\end{subfigure}
	%\caption{Графики значений элементов теоретической и экспериментальной матриц вероятностей ошибок}
	%\label{fig:test}
\end{figure}
~\\

\end{spacing}
\end{document}